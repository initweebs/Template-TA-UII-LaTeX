%!TEX root = ./template-skripsi.tex
%-------------------------------------------------------------------------------
% 								BAB I
% 							LATAR BELAKANG
%-------------------------------------------------------------------------------

\chapter{LATAR BELAKANG}

\section{Latar Belakang Masalah}
Dewasa ini banyak produsen dari suatu produk yang mulai menjadikan hal-hal non-teknis seperti estetika, afeksi dan kenyamanan sebagai salah satu pertimbangan utama dari produk yang mereka buat. Hal ini dikarenakan pihak produsen sendiri ingin membuat suatu produk yang dapat menimbulkan ikatan emosional dengan penggunannya. Salah satu aspek agar ikatan emosional antara produk dengan pengguna adalah dengan sentuhan.\\
\indent Tekstur dari suatu permukaan produk mempunyai peran yang penting dalam terciptanya kesan emosional dari suatu produk melalui sentuhan. Ini dikarenakan tekstur merupakan suatu fitur penting yang digunakan manusia untuk memperoleh informasi melalui sentuhan. Pihak produsen atau pengembang berusaha untuk membuat produk mereka memiliki tekstur yang dapat menimbulkan kesan emosional pada pengguna seperti yang mereka inginkan. \\
\indent Saat ini perancangan dan riset dari tekstur permukaan yang dapat menimbulkan kesan emosional tertentu pada manusia menggunakan metode trial and error. Hal ini tentunya sangat merugikan produsen mengingat memakan banyak waktu dan biaya. Ini memicu banyak penelitian terkait pendekatan saintifik untuk menentukan tekstur suatu permukaan yang dapat menimbulkan kesan emosional tertentu pada manusia. Salah satu alternatif yang ditawarkan oleh penulis adalah dengan mengidentifikasi karakteristik aspek afektif dari permukaan menggunakan fitur berbasis fourier transform. 

\section{Rumusan Masalah}
\indent Berdasarkan latar belakang yang telah dipaparkan di atas maka diperoleh rumusan masalah sebagai berikut:
\begin{enumerate}
	\item Bagaimana aspek afektif dari suatu permukaan direprsentasikan dengan fitur berbasis Fourier Transform?
	\item Bagaimana aspek afektif dari suatu permukaan?
\end{enumerate}


\section{Batasan Masalah}
\indent Berdasarkan rumusan masalah tersebut pada penelitian ini akan dilakukan pembatasan masalah sebagai berikut:
\begin{enumerate}
\item Penelitian ini menggunakan subjek uji yaitu WNI dengan rentang usia dari 20 tahun sampai dengan 25 tahun.
\item Jumlah fitur yang digunakan untuk menganalisis karakter adalah 109 fitur.
\item Penyusunan terminologi yang berkaitan dengan aspek afektif berasal dari perbandingan hasil pengujian awal dengan referensi yang sudah ada.
\item Terminologi afeksi yang dipakai adalah yang tercantum dari kamus besar bahasa indonesia(KBBI).
\item Penelitian ini hanya mempelajari aspek afeksi dari sentuhan manusia pada sebuah permukaan saja.
\item Penelitian ini dibatasi hanya sampai dengan pembuatan model dan analisis hubungan antara fitur berbasis Fourier Transform dengan afeksi yang dihasilkan dari sentuhan saja.

\end{enumerate}


\section{Tujuan Penelitian}
\indent Berdasarkan rumusan masalah di atas, maka penelitian ini dilakukan dengan tujuan sebagai berikut:
\begin{enumerate}
	\item Mengetahui karakteristik aspek afektif dari suatu permukaan menggunakan fitur berbasis Fourier Transform.
	\item Mengetahui macam-macam aspek afektif yang ada pada suatu permukaan.
	\item Mengetahui hubungan antara fitur berbasis Fourier Transform dengan aspek afektif dari suatu permukaan.
\end{enumerate}


\section{Manfaat Penelitian}
\indent Terdapat beberapa manfaat dari penelitian ini untuk beberapa pihak seperti peneliti,kampus dan industri.

\subsection{Manfaat untuk Peneliti}
\begin{enumerate}[a.]
	\item Mengetahui aspek afektif dari suatu permukaan dengan metode yang lebih akurat.
	\item Mengetahui hubungan antara aspek afektif dari sentuhan dengan fitur berbasis Fourier Transform.
	\item Mengetahui karakteristik permukaan yang disukai dan tidak disukai oleh seseorang dengan fitur berbasis Fourier Transform.
\end{enumerate}

\subsection{Manfaat untuk Kampus}
\begin{enumerate}[a.]
	\item Memperoleh perspektif baru mengenai \emph{affective engineering design.}
	\item Mengetahui potensi research area disiplin ilmu teknik mesin yang berkaitan dengan disiplin ilmu lain.
\end{enumerate}

\subsection{Manfaat untuk Industri}
\begin{enumerate}[a.]
	\item Mengetahui karakteristik permukaan yang berpotensi untuk disukai/digemari konsumen.
	\item Memiliki acuan yang untuk merekayasa suatu permukaan agar produk dapat menimbulkan ikatan emosional kepada konsumen.
\end{enumerate}

\section{Sistematika Penulisan}
\noindent
\textbf{BAB I : PENDAHULUAN}

Pada bab ini dijelaskan latar belakang, rumusan masalah, batasan, tujuan, manfaat, keaslian penelitian, dan sistematika penulisan.\\

\noindent
\textbf{BAB II : TINJAUAN PUSTAKA DAN LANDASAN TEORI}

Pada bab ini dijelaskan teori-teori dan penelitian terdahulu yang digunakan sebagai acuan dan dasar dalam penelitian.\\

\noindent
\textbf{BAB III : METODOLOGI PENELITIAN}

Pada bab ini dijelaskan metode yang digunakan dalam penelitian meliputi langkah kerja, pertanyaan penilitian, alat dan bahan, serta tahapan dan alur penelitian.\\

\noindent
\textbf{BAB IV : HASIL DAN PEMBAHASAN}

Pada bab ini dijelaskan hasil penelitian dan pembahasannya.\\

\noindent
\textbf{BAB V : KESIMPULAN DAN SARAN}

Pada bab ini ditulis kesimpulan akhir dari penelitian dan saran untuk pengembangan penelitian selanjutnya.\\

% Baris ini digunakan untuk membantu dalam melakukan sitasi
% Karena diapit dengan comment, maka baris ini akan diabaikan
% oleh compiler LaTeX.
\begin{comment}
\bibliography{daftar-pustaka}
\end{comment}
